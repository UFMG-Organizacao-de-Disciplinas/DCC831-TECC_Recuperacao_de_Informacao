
\title[Exploring IR Techniques Through PA 2]{Exploring Information Retrieval Techniques Through Programming Assignment 2} % \title[short title]{Full title}

\begin{abstract}
  This report documents the implementation of an indexer and query processor for a search engine system. The solution handles large-scale document processing within strict memory constraints using parallelization and efficient data structures. The system includes stopword removal, stemming, conjunctive document-at-a-time matching, and supports both TFIDF and BM25 ranking functions. Empirical evaluation shows (not so) efficient indexing of 4.6M Wikipedia entities and effective query processing.
\end{abstract}

%% Keywords. The author(s) should pick words that accurately describe the work being presented. Separate the keywords with commas.
\keywords{Information Retrieval, Indexer, Python, Query Processor, TFIDF, BM25, Stopword Removal, Stemming, Parallelization}

%% A "teaser" image appears between the author and affiliation information and the body of the document, and typically spans the page.
% \begin{teaserfigure}
%   \includegraphics[width=\textwidth]{Web_Crawler}
%   \caption{Web Crawler Logo}
%   \Description{An AI generated image of a logo for the web crawler project. The logo is a stylized representation of a spider for the crawling meaning, also related to the World Wide Web}
%   \label{fig:teaser}
% \end{teaserfigure}

\received{02 June 2025}
% \received[revised]{28 April 2025}
% \received[accepted]{5 May 2025}

%% This command processes the author and affiliation and title information and builds the first part of the formatted document.
\maketitle

