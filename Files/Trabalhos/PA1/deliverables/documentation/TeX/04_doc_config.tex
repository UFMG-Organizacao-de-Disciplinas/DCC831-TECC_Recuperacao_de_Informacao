
\title{Information Retrieval - Programming Assignment 1} % \title[short title]{Full title}

\begin{abstract}
  This article is summary of the implementation of the first programming assignment of the Information Retrieval course at the Federal University of Minas Gerais (UFMG). The assignment consists of programming a web crawler using Python 3 to scrape 100.000 unique pages efficiently, respecting certaing policies. The crawler must be distributed, using the multiprocessing library to take advantage of multiple CPU cores. After crawling, the pages must be stored into a WARC file, which is a standard format for archiving web pages, and zipped to save space.
\end{abstract}

%% Keywords. The author(s) should pick words that accurately describe the work being presented. Separate the keywords with commas.
\keywords{Information Retrieval, Web Crawler, Python, WARC, Multiprocessing}

%% A "teaser" image appears between the author and affiliation information and the body of the document, and typically spans the page.
\begin{teaserfigure}
  \includegraphics[width=\textwidth]{Web_Crawler}
  \caption{PA1 Web Crawler JVFD}
  \Description{An AI generated image of a logo for the web crawler project. The logo is a stylized representation of a spider for the crawling meaning, also related to the World Wide Web}
  \label{fig:teaser}
\end{teaserfigure}

\received{31 March 2025}
\received[revised]{28 April 2025}
% \received[accepted]{5 May 2025}

%% This command processes the author and affiliation and title information and builds the first part of the formatted document.
\maketitle

