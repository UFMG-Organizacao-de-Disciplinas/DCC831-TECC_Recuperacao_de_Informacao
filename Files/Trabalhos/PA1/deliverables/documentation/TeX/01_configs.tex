\documentclass[sigconf]{acmart}

%% \BibTeX command to typeset BibTeX logo in the docs
\AtBeginDocument{\providecommand\BibTeX{{Bib\TeX}}}

%% Rights management information.  This information is sent to you when you complete the rights form.  These commands have SAMPLE values in them; it is your responsibility as an author to replace the commands and values with those provided to you when you complete the rights form.
\setcopyright{acmlicensed}
\copyrightyear{2018}
\acmYear{2018}
\acmDOI{XXXXXXX.XXXXXXX}

%% These commands are for a PROCEEDINGS abstract or paper.
\acmConference[Conference acronym 'XX]{Make sure to enter the correct conference title from your rights confirmation email}{June 03--05, 2018}{Woodstock, NY}

%% Uncomment \acmBooktitle if the title of the proceedings is different from ``Proceedings of ...''!
%%\acmBooktitle{Woodstock '18: ACM Symposium on Neural Gaze Detection, June 03--05, 2018, Woodstock, NY}

\acmISBN{978-1-4503-XXXX-X/2018/06}

%% Submission ID: Use this when submitting an article to a sponsored event. You'll receive a unique submission ID from the organizers of the event, and this ID should be used as the parameter to this command.
%%\acmSubmissionID{123-A56-BU3}

%% For managing citations, it is recommended to use bibliography files in BibTeX format.

%% You can then either use BibTeX with the ACM-Reference-Format style, or BibLaTeX with the acmnumeric or acmauthoryear sytles, that include support for advanced citation of software artefact from the biblatex-software package, also separately available on CTAN.

%% Look at the sample-*-biblatex.tex files for templates showcasing the biblatex styles.

%% The majority of ACM publications use numbered citations and references.  The command \citestyle{authoryear} switches to the "author year" style.

%% If you are preparing content for an event sponsored by ACM SIGGRAPH, you must use the "author year" style of citations and references.
%% Uncommenting the next command will enable that style.
%%\citestyle{acmauthoryear}

%% end of the preamble, start of the body of the document source.